\chapter[Протокол получения данных]{Протокол получения данных}
\label{chap:acquisition_protocol}

Для правильной работы программы требуются некоторые основные шаги, касающиеся
процедуры сбора данных. Некоторые процедуры сбора данных также следует применять
в более общем смысле, чтобы получать более точные результаты и проводить
эффективные обследования. Подробный анализ и описание гравитационных сетей см.,
например (\cite{lambert_nano_1977, torge_1980}). Требования и рекомендации по
высококачественным проектам обследования можно найти в \cite{seigel_1995}.

\section[Геометрия петли и номенклатура]{Геометрия петли и номенклатура}
\label{sec:loop_geometries_and_nomenclature}

\begin{itemize}
    \item Отдельные пункты всегда должны иметь один и тот же идентификационный
    номер (название), независимо от их статуса (опорный/рядовой
    пункт). Если он изменен, его можно модифицировать вручную, используя процедуру
    поиска/замены в файле необработанных данных в текстовом редакторе.
    
    \item Различные опорные пункты в рамках одной схемы обработки до сих пор не
    обрабатываются, но программа может запускаться последовательно для каждого
    подмножества данных.
    
    \item Предпочтительно, но не обязательно, чтобы опорный пункт имел другое
    название, чем пункт цикла (если он соответствует тому же местоположению).
    
\end{itemize}

\section[Транспортировка]{Транспортировка}
\label{sec:transportation}

\begin{itemize}
    \item CG5 очень чувствителен к транспортировке: на короткие расстояния его
    могут переносить два оператора, которые держат рейку, на которую за ручки
    подвешен мешок с гравиметром. На больших расстояниях он должен быть в лучшем
    случае изолирован от вибрации автомобиля.

    \item Транспортировка обычно подразумевает так называемый кратковременный
    "транспортный дрейф", и стабилизация пружины после достижения станции может
    занять некоторое время. Этап выбора данных в pyGrav позволяет выбрать данные
    после стабилизации прибора.

\end{itemize}

\section[Настройка прибора]{Настройка прибора}
\label{sec:setting_up_the_instrument}

\begin{itemize}
    \item Поскольку вертикальные перемещения сильно влияют на изменение силы
    тяжести ($\approx$0,3 мкгал/мм при номинальном градиенте свободного воздуха),
    станции должны быть оборудованы бетонными опорами, чтобы ограничить
    последствия мягкого грунта или других неустойчивых элементов.В качестве
    альтернативы можно было бы использовать приводные стержни для ограничения
    роли усадки/набухания грунта. Для каждой съемки (в режиме замедленной
    съемки) прибор должен располагаться точно в одной и той же точке и
    удерживаться под одним и тем же азимутом, например, благодаря винтам,
    воткнутым в бетонную стойку. Высоту относительно стойки следует поддерживать
    постоянной с помощью латунного кольца, прикрепленного к одной ножке штатива
    прибора, как предложил \cite{montgomery_1971}.
    
    \item CG5 и его штатив должны быть защищены от ветра и прямых солнечных
    лучей, например, мусорным баком, покрытым изоляционным материалом, или
    соответствующим зонтиком.
    
\end{itemize}

\section[Сбор данных]{Сбор данных}
\label{sec:data_acquisition}

\begin{itemize}
    \item В настоящее время нет возможности применить различные поправки к
    приливам в рамках одной и той же съемки. Следовательно, если съемка состоит
    из станций, удаленных друг от друга на большое расстояние (несколько
    десятков км, не рекомендуется для высокоточных покадровых гравитационных
    съемок), пользователь должен позаботиться о том, чтобы ввести правильные
    местоположения станций в CG5 на местах, чтобы можно было применить поправку
    CG5 на земные приливы.
    
    \item Длительность: CG5 производит выборку с частотой 6 Гц и усреднение по
    заданной пользователем длительности для получения одного единственного
    измерения. \cite{merlet_micro-gravity_2008} показали в своем исследовании, что
    отклонение по Аллану их CG5 достигло 1 мкгал через 40 с, до минимума в 0,8
    мкгал через 85 с, но еще больше увеличилось из-за влияния приливов.
    \cite{gettings_techniques_2008} также показывают пример из необработанных
    данных за 1 секунду (их рисунок 2), где среднее значение сходится примерно
    через 40 секунд.  Кроме того, по истечении 60 секунд стандартная ошибка на
    экране дисплея CG5, работающего в полевых условиях, всегда, по-видимому,
    сходится. Хранение данных за 60 секунд часто позволяет сделать динамичный и
    быстрый вывод о стабильности прибора (см., например, \cite{hector_water_2015}).

    \item Идентификация стабилизации: Существует столько вариантов, сколько
    требуется пользователям и научным целям для определения того, является ли
    полученный временной ряд достаточно длинным (стабилизированным) или нет.
    Строгая процедура идентификации сигналов малой амплитуды при покадровых
    исследованиях может быть следующей: на станции проводится первая серия из 5
    измерений, пока операторы находятся на расстоянии нескольких метров от
    прибора, прежде чем в первый раз проверяется выравнивание. Затем проводится
    еще один набор измерений, и операторы приходят проверять стабильность работы
    прибора примерно каждые 5 минут. При любой проверке, если наклоны выходят за
    пределы диапазона 0 ± 5 дюймов (или любого другого выбранного порогового
    значения), прибор снова выравнивается. Это следует за \cite{merlet_micro-gravity_2008},
    которые обнаружили, что внутренняя коррекция наклона CG5 является точной на
    уровне 1 мкгал при ± 20‘. Кроме того, им удалось сохранить наклоны в
    пределах 0 ± 3 ’ в условиях их внутреннего и стабильного пирса. Измерения
    можно считать выполненными (и гравиметр стабильным), если выполнены все
    следующие критерии: - выполнено минимум 10 соответствующих измерений; -
    изменения силы тяжести составляют 3 мкгал или менее в течение 5
    последовательных измерений; - в 5 последних измерениях нет видимого дрейфа
    (дрейф <1 мкгал/мн).. Все это означает, что долгосрочный дрейф CG5 внутренне
    корректируется прибором, скажем, при остаточном дрейфе < 100 мкгал/сут
    (около 1 мкгал/15млн).
    
\end{itemize}