\chapter[Пример для проверки]{Пример для проверки}
\label{chap:test-case}

В этом разделе подробно описаны два способа обработки тестового примера. Первый
использует графический интерфейс \pg{} для быстрой и удобной обработки данных
микрогравиметрии. Второй использует сценарий Python, который вызывает функции
\pg{}. Это больше всего подходит для тех, кто заинтересован в разработке \pg{} и
добавлении новых функциональных возможностей.

\section[Использование графического интерфейса pyGrav]{Использование графического интерфейса pyGrav}
\label{sec:using_pygrav_gui}

Тестовый пример доступен в каталоге \verb|test_case|. Это набор из 4 съемок,
проведенных в ходе Западноафриканской кампании в небольшом водосборном бассейне
на севере Бенина. Каждая съемка обычно состоит из четырех петель, охватывающих в
общей сложности 13 пунктов и базовую станцию (станция \textnumero{}~1). 

Чтобы обработать этот набор данных из файла необработанных данных, выполните
следующие действия:
\begin{itemize}
    \item Запустите \pg{}: запустите файл Python \verb|pyGrav_main.py| используя
    python(x,y), например, в Windows, или выполнив \verb|python pyGrav_main.py| в
    оболочке Linux.

    \item Запустите проект: выберите подкаталоги \verb|input_data/raw| и \verb|output_data| из
    каталога \verb|test_case|.

    \item Выберите \textbf{Load raw data} в меню \textbf{File}: откройте \verb|Atest_Raw_data.txt|

    \item Выберите \textbf{Load Survey dates file} и загрузите
    \verb|Atest_start_end_dates.txt|, или в качестве альтернативы, выберите одну
    съёмку путём введения начальных и конечных дат (см. файл
    \verb|Atest_start_end_dates.txt|). Номер базовой станции равен 1.

    \item Обработка приливов: Выберите \textbf{Tide corrections} в меню
    \textbf{Process} и выберите \textbf{Use CG5 tide corrections} или в качестве
    альтернативы \textbf{Use Synthetic tides from predict} (если установлено,
    см. раздел~\ref{chap:installing_external_programs}): Широта = 9.742; Долгота
    = 1.606; Высота над уровнем моря = 450. В этом случае выберите \textbf{Load
    Tidal parameters} и загрузите \verb|tide_param.txt| из папки
    \verb|input_data|. Этот список параметров был получен на основе приливного
    анализа сверхпроводящего гравиметра, доступного на месте исследования, и
    следовательно, эмпирически включает океаническую нагрузку.  В качестве
    последнего варианта, выберите \textbf{Use Synthetic tides from Agnew}:
    Широта = 9.742; Долгота = 1.606; Высота над уровнем моря = 450.

    \item Обработка океанической нагрузки: выберите \textbf{Ocean Loading correction} в
    меню \textbf{Process} и выберите файл \verb|oceantidal.txt| из папки \verb|input_data|. Этот файл
    был получен из \url{http://holt.oso.chalmers.se/loading/} с широтой = 9.742;
    долготой = 1.606; высотой над уровнем моря = 450.

    \item Выбор данных: выберите \textbf{Data selection} в меню \textbf{Process}.
    \begin{itemize}
        \item Перейдите в дерево на левой панели, чтобы проверить временные ряды
        для некоторых пунктах, щелкнув на пункт в петле. Сначала все измерения
        проверяются и отображаются синим цветом в окнах графика (правая панель).
        
        \item Попробуйте снять галочки с некоторых данных в табличном
        представлении. Графики обновляются при повторном нажатии на пункт в
        дереве (левая панель) или при нажатии кнопки \textbf{update plots} на верхней
        панели. Черные точки появляются там, где данные не отмечены. Эти
        измерения не будут сохранены для компенсации сети. Горизонтальная синяя
        линия на графике силы тяжести -- это среднее значение силы тяжести для
        отмеченных значений.

        \item Автоматический выбор. Заполните следующие критерии автоматического
        выбора и выберите \textbf{apply to all data}:
        \begin{itemize}
            \item \textbf{auto uncheck tilts >}: 5
            \item \textbf{auto uncheck g >}: 4
            \item \textbf{auto uncheck SD >}: 20
            \item \textbf{auto uncheck dur <>}: 60
        \end{itemize}

        \item Просмотрите подборку, чтобы завершить фактический выбор (при
        необходимости).

        \item Нажмите кнопку OK. Это действие мало что дает, но может быть
        важным: оно используется для проверки того, что на некоторых станциях
        данные не выбраны, и в этом случае они удаляются. Это может произойти с
        помощью автоматического выбора.
        
    \end{itemize}
    
    \item Теперь обработанные данные можно сохранить: в меню \textbf{File}
    выберите \textbf{Save processed data}. В папке \verb|output_data| будут
    созданы подкаталоги, имена которых являются именами съёмок и которые
    содержат файлы петель, аналогичные файлам CGxTool 'c', с дополнительным
    столбцом 0 или 1, указывающим, следует ли сохранить станцию или нет. Вся
    иерархия данных записана в файле \verb|gravity_data_hierarchy.txt| в папке
    \verb|output_data|. Этот файл можно открыть в другом проекте для загрузки
    обработанного набора данных (выбрав пункт загрузить обработанные данные в
    меню Файл).
    
    \item Коррекция дрейфа: выберите \textbf{Drift Adjustment} в меню
    \textbf{Process} и \textbf{Use datum-free least-square inversion} без
    привязки к данным с параметрами:
    \begin{itemize}
        \item \textbf{Temporal drift polynomial?}: 1
        \item \textbf{Temperature drift polynomial?}: 0
        \item \textbf{SD factor to data}: 1
        \item \textbf{SD add to data (mgal)}: 0.005
        \item \textbf{Significance level for global model test}: 0.05 (т.~е. 5\%)
        \item \textbf{Write Output Files (y/n)?}: y
    \end{itemize}

    \item Теперь дрейф скорректирован и можно проверить выходные файлы, чтобы
    увидеть, удовлетворяет ли вывод дрейфа или нет. В каждой подпапке съёмки 
    (которая была создана, если она ранее не была создана функцией \textbf{Save
    processed data}) из папки \verb|output-data| два файла, начинающихся с
    \verb|LSresults|, отображают результаты единичных разностей для первого и полную
    процедуру LS (ввод, выходные данные и полная проверка) для второго.

    \item Сохранение единичных разностей: выберите \textbf{Save simple difference} в меню
    \textbf{File}. Как и в случае с функцией \textbf{Save processed data}, это создаст
    файлы единичных разностей во вложенных папках папки \verb|output_data| вместе с файлом
    иерархии, начинающимся с \verb|simple_diff_data_hierarchy...txt|

    \item Вычисление двойной разности: выберите \textbf{Compute double difference} в меню
    \textbf{Process}. Выберите опорную съёмку, например, первую. Затем выберите
    \textbf{Classic double difference}.

    \item Теперь возможно сохранить двойные разности: выберите \textbf{Save
    double difference} в меню \textbf{File}. Это создаст файлы \verb|gravity| и
    \verb|SD double difference| в каталоге \verb|output_data|. Доступны два
    формата (даты -- пункты или пункты -- даты).

    \item Альтернативой здесь является загрузка уже обработанных данных (в
    данном случае, уже выбранных данных), и для сравнения результатов
    уравниваний дрейфа, единичных и двойных разностей с выбранными вами
    результатами. В этом случае начните с изменения пути к данным в первой
    строке файла \verb|gravity_data_hierarchy.txt| в каталоге
    \verb|input_data/preprocessed|. Затем выберите пункт \textbf{Load processed
    data} в меню \textbf{File} и выберите файл \verb|gravity_data_hierarchy.txt| в
    каталоге \verb|input_data/preprocessed|. Этот файл является файлом иерархии,
    описывающим, где найти файлы петли в каждом каталоге съёмки (как указано в
    функции \textbf{Save processed data}).
    
\end{itemize}

\section[Использование сценариев python и объектов pyGrav]{Использование сценариев python и объектов pyGrav}
\label{sec:using_a_python_script_and_pygrav_objects}

Другой способ использования функций \pg{} (в \verb|data_objects.py| файл)
заключается в написании сценария на Python и последовательном вызове этих
функций. Это особенно подходит, если кто-то хочет добавить функцию в основной
код и протестировать ее, не проходя через все программирование с графическим
интерфейсом. Как только функция настроена должным образом, дополнительные
кнопки/действия графического интерфейса могут быть легко добавлены в
\verb|pyGrav_main.py| путем адаптации существующего кода.

Пример сценария (\verb|example_script.py|) можно найти в каталоге
\verb|main_code|, и он должен выполняться как скрипт на Python. Цель скрипта --
загрузить необработанные данные, извлечь отдельные съёмки на основе дат начала и
окончания, выбрать данные для сохранения на основе простых пороговых значений,
уравнять дрейф для каждой съёмки, и в конечном итоге вычислить как единичные,
так и двойные разности. Комментируя/раскомментируя некоторые части кода, также
можно загрузить уже выбранные и упорядоченные данные.