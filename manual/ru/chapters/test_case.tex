\chapter[Пример для проверки]{Пример для проверки}
\label{chap:test-case}

В этом разделе подробно описаны два способа обработки тестового примера. Первый
использует графический интерфейс pyGrav для быстрой и удобной обработки данных о
микрогравитации. Второй использует скрипт на python, который вызывает функции
pyGrav. Это больше всего подходит для тех, кто заинтересован в разработке pyGrav
и добавлении новых функциональных возможностей.

\section[Использование графического интерфейса pyGrav]{Использование графического интерфейса pyGrav}
\label{sec:using_pygrav_gui}

Тестовый пример доступен в каталоге \verb|test_case|. Это набор из 4 съемок,
проведенных в ходе Западноафриканской кампании в небольшом водосборном бассейне
на севере Бенина. Каждая съемка обычно состоит из четырех циклов, охватывающих в
общей сложности 13 станций и базовую станцию (станция №1). 

Чтобы обработать этот набор данных из файла необработанных данных, выполните
следующие действия:
\begin{itemize}
    \item Запустите pyGrav: запустите файл python \verb|pyGrav_main.py| используя
    python(x,y), например, в Windows, или выполнив \verb|“python pyGrav_main.py”| в
    оболочке Linux.

    \item Запустите проект: выберите подкаталоги \verb|input_data/raw| и \verb|output_data| из
    каталога \verb|test_case|.

    \item Выберите \textbf{Load raw data} в меню \textbf{File}: откройте \verb|Atest_Raw_data.txt|

    \item Выберите \textbf{Load Survey dates file} и загрузите
    \verb|Atest_start_end_dates.txt|, или в качестве альтернативы, выберите одну
    съёмку путём введения начальных и конечных дат (см. файл
    \verb|Atest_start_end_dates.txt|). Номер базовой станции равен 1.

    \item Обработка приливов: Выберите коррекцию приливов в меню Процесс и
    выберите Использовать коррекцию приливов CG5 или в качестве альтернативы
    использовать синтетические приливы из predict (если установлено, см. раздел
    8): Широта = 9.742; Долгота = 1.606 ; Высота над уровнем моря = 450. В этом
    случае выберите Параметры прилива нагрузки и загрузите \verb|tide_param.txt| из
    папки \verb|input_data|. Этот список параметров был получен на основе приливного
    анализа сверхпроводящего гравиметра, доступного на месте исследования, и,
    следовательно, эмпирически включает океаническую нагрузку. Последний
    вариант, выберите Использовать синтетические приливы из Agnew: Широта =
    9.742; Долгота = 1.606 ; Высота над уровнем моря = 450.

    \item Обработать загрузку в океане: выберите Исправление загрузки в океане в
    меню Процесс и выберите файл \verb|oceantidal.txt| из папки \verb|input_data|. Этот файл
    был получен из \url{http://holt.oso.chalmers.se/loading/} с широтой = 9.742;
    долготой = 1.606 ; высотой над уровнем моря = 450.

    \item Выбор данных: выберите пункт Выбор данных в меню Процесса.
    \begin{itemize}
        \item Перейдите в дерево на левой панели, чтобы проверить временные ряды
        для некоторых станций, щелкнув станцию в цикле. Сначала все измерения
        проверяются и отображаются синим цветом в окнах графика (правая панель).
        
        \item Попробуйте снять галочки с некоторых данных в табличном
        представлении. Графики обновляются при повторном нажатии на станцию в
        дереве (левая панель) или при нажатии кнопки обновить графики на верхней
        панели. Черные точки появляются там, где данные не отмечены. Эти
        измерения не будут сохранены для сетевой компенсации. Горизонтальная
        синяя линия на графике гравитации - это среднее значение гравитации для
        отмеченных значений.

        \item Автоматический выбор. Заполните следующие критерии автоматического
        выбора и выберите применить ко всем данным:
        \begin{itemize}
            \item автоматическое снятие флажка наклоны > : 5
            \item автоматическое снятие флажка g > : 4
            \item автоматическое снятие флажка SD > : 20
            \item автоматическое снятие флажка dur <>: 60
        \end{itemize}

        \item Просмотрите подборку, чтобы завершить фактический выбор (при
        необходимости).

        \item Нажмите кнопку OK. Это действие мало что дает, но может быть
        важным: оно используется для проверки того, что на некоторых станциях
        данные не выбраны, и в этом случае они удаляются. Это может произойти с
        помощью автоматического выбора.
        
    \end{itemize}
    
    \item Теперь можно сохранить обработанные данные: в меню Файл выберите
    Сохранить обработанные данные. В этой папке \verb|output_data| будут созданы
    подкаталоги, имена которых являются именами опроса и которые содержат файлы
    цикла, аналогичные файлам CGxTool ‘c’, с дополнительным столбцом 0 или 1,
    указывающим, следует ли сохранить станцию или нет. Вся иерархия данных
    записана в файле \verb|gravity_data_hierarchy.txt| в папке
    \verb|output_data|. Этот файл можно открыть в другом проекте для загрузки
    обработанного набора данных (выбрав пункт загрузить обработанные данные в
    меню Файл).
    
    \item Коррекция дрейфа: выберите корректировку дрейфа в меню процесса и
    используйте инверсию наименьших квадратов без привязки к данным с
    параметрами:
    \begin{itemize}
        \item Полином временного дрейфа ?: 1
        \item полином температурного дрейфа?: 0
        \item Коэффициент SD к данным: 1
        \item SD добавить к данным (мгал): 0,005
        \item Уровень значимости для теста глобальной модели: 0,05 (т.е. 5%)
        \item Записать выходные файлы (y/n)?: y
    \end{itemize}

    \item Дрейф теперь скорректирован, и можно проверить выходные файлы, чтобы
    увидеть, удовлетворяет ли поиск дрейфа или нет. В каждой подпапке опроса
    (которая была создана, если она еще не была создана функцией сохранения
    обработанных данных) из папки выходных данных два файла, начинающихся с
    LSresults, отображают простые результаты различий для первого и полную
    процедуру LS (ввод, выходные данные и глобальный тест) для второго..

    \item Сохранить простые различия: выберите Сохранить простые различия в меню
    Файл. Как и в случае с функцией сохранения обработанных данных, это создаст
    простые файлы различий во вложенных папках папки \verb|output_data| вместе с файлом
    иерархии, начинающимся с \verb|simple_diff_data_hierarchy...txt|

    \item Вычислить двойную разницу: выберите "Вычислить двойную разницу" в меню
    "Процесс". Выберите эталонный опрос, например, первый. Затем выберите
    классические двойные различия.

    \item Теперь возможно сохранить двойные различия: выберите Сохранить двойные
    различия в меню Файл. Это создаст файлы \verb|gravity| и \verb|SD double difference| в
    каталоге \verb|output_data|. Доступны два формата (даты против станций или станции
    против даты).

    \item Альтернативой здесь является загрузка уже обработанных данных (в
    данном случае, уже выбранных данных) и сравнение результатов корректировки
    дрейфа, простых и двойных различий с результатами, полученными в результате
    вашего собственного выбора. В этом случае начните с изменения пути к данным
    в первой строке \verb|gravity_data_hierarchy.txt| файл в каталоге
    \verb|input_data/preprocessed|. Затем выберите пункт Загрузить обработанные данные
    в меню Файл и выберите \verb|gravity_data_hierarchy.txt| файл в каталоге
    \verb|input_data/preprocessed|. Этот файл является файлом иерархии, описывающим,
    где найти файлы цикла в каждом каталоге опроса (как указано в функции
    сохранения обработанных данных.
    
\end{itemize}

\section[Использование сценариев python и объектов pyGrav]{Использование сценариев python и объектов pyGrav}
\label{sec:using_a_python_script_and_pygrav_objects}

Другой способ использования функций pyGrav (в \verb|data_objects.py| файл) заключается
в написании скрипта на python и последовательном вызове этих функций. Это
особенно подходит, если кто-то хочет добавить функцию в основной код и
протестировать ее, не проходя через все программирование с графическим
интерфейсом. Как только функция настроена должным образом, дополнительные кнопки
/ действия графического интерфейса могут быть легко добавлены в \verb|pyGrav_main.py|
путем адаптации существующего кода.

Пример скрипта (\verb|example_script.py|) можно найти в каталоге \verb|main_code| и должен
выполняться как скрипт на python. Цель скрипта - загрузить необработанные
данные, извлечь отдельные опросы на основе дат начала и окончания, выбрать
данные для сохранения на основе простых пороговых значений, скорректировать
отклонения для каждого опроса и в конечном итоге вычислить как простые, так и
двойные различия. Комментируя/раскомментируя некоторые части кода, также можно
загрузить уже выбранные и упорядоченные данные.
