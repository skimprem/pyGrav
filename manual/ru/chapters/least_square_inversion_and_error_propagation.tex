\chapter[Среднеквадратическая инверсия и распределение ошибок]{Среднеквадратическая инверсия и распределение ошибок}
\label{chap:least-square_inversion_and_error_propagation}

Гравитационные наблюдения на станции представляют собой временные ряды из
нескольких гравитационных измерений, каждое измерение является средним значением
нескольких выборок. Например, типичные результаты измерения CG-5 получены в
результате сбора образцов в течение более минуты (наилучшим считается 85 секунд
\cite{merlet_micro-gravity_2008}) при частоте 6 Гц. Таким образом, стандартная погрешность
(SE) при каждом измерении составляет:
\begin{equation}
    SE = \frac{\sigma}{\sqrt{n}}
\end{equation}
где $\sigma$~-- стандартное отклонение измерения, а $n$~-- количество выборок.

После отбора данных из оставшихся временных рядов выводятся одно наблюдение $l_i$
и стандартное отклонение $\sigma_i$ для каждой станции $i$ с использованием средних,
взвешенных по дисперсии:
\begin{equation}
    l_i = \frac{\sum_{j=1}^{n}\frac{1}{SE^2_j}}{\sum_{j=1}{n}\frac{1}{SE^2_j}}
\end{equation}
\begin{equation}
    \sigma_i = \frac{1}{\sum_{j=1}^{n}\frac{1}{SE_j^2}}
\end{equation}

Стандартное отклонение $\sigma_{ij}$ для наблюдения относительной силы тяжести
$\Delta l_{ij}$ между станциями $i$ и $j$ определяется как
\begin{equation}
    \sigma_{ij} = \sqrt{\sigma_i^2 + \sigma_j^2},
\end{equation}
где $\Delta l_{ij}$ определяется как
\begin{equation}
    \Delta l_{ij} + \upsilon_{ij} = g_j + g_i + \sum_{T=1}^{m} a_T\left(t_j - t_i\right)^T + \sum_{k=1}^{m} b_k \left(t_j - t_i\right)^k
\end{equation}

Здесь $\upsilon_{ij}$~-- невязки, $g_i$~-- значение силы тяжести на станции $i$,
$m$~-- степень полинома коэффициентов $a_T$ для дрейфа гравиметра и $n$~--
степень полинома коэффициентов $b_k$ для температурного дрейфа. Таким образом,
система уравнений является
\begin{equation}
    \mathbf{L}^{\mathbf{b}} + \mathbf{V} = \mathbf{A}\mathbf{X},
\end{equation}
где $\mathbf{L}^{\mathbf{b}}$ вектор, содержащий $n$ наблюдений относительной
силы тяжести, $\mathbf{V}$~-- вектор с $n$ остатками, $\mathbf{A}$~-- расчетная
матрица и $\mathbf{X}$~-- вектор $u$ неизвестных (значений силы тяжести и
параметров дрейфа). Чтобы получить решение $\mathbf{X}$, необходимо удерживать
фиксированным по крайней мере одно значение силы тяжести во время настройки (так
называемое исходное значение силы тяжести).  Здесь это делается путем добавления
наблюдений за абсолютной гравитацией (обычно равной нулю) на базовой станции:
\begin{equation}
    \mathbf{L}^{\mathbf{b}}_{\mathbf{g}} + \mathbf{V}_{\mathbf{g}} =
    \mathbf{A}_{\mathbf{g}\mathbf{X}} =
    \left[\mathbf{I}0\right]
    \left[
        \begin{aligned}
            \mathbf{X}_{\mathbf{g}}\\
            \mathbf{X}_{\mathbf{I}}
        \end{aligned}
    \right]
\end{equation}

Введя вектор $S$ с $k$ (числом значений силы тяжести), первые значения которых равны
единице, а последние значения $u-k$ равны нулю, что удовлетворяет:
\begin{equation}
    \mathbf{S}^{\mathbf{T}}\mathbf{X}=0
\end{equation}
можно найти решение по методу наименьших квадратов для $X$:
\begin{equation}
    \mathbf{X} = \left(\mathbf{A}^{\mathbf{T}}\mathbf{P}\mathbf{A} + \mathbf{S}\mathbf{S}^{\mathbf{T}}\right)^{-1}\mathbf{A}^{\mathbf{T}}\mathbf{P}\mathbf{L}^{\mathbf{b}}
\end{equation}

Матрица весов $\mathbf{P}$ состоит из членов, обратных дисперсии для наблюдений.

Апостериорная дисперсия единичного веса ($\sigma^2_0$) задается формулой
\begin{equation}
    \sigma^2_0 = \frac{\mathbf{V}^{\mathbf{T}}\mathbf{P}\mathbf{V}}{n+1-u}
\end{equation}
и апостериорная ковариационная матрица $\mathbf{S}_X$ задается путем
распространения ковариации:
\begin{equation}
    \mathbf{S}_X = \sigma^2_0\left(\mathbf{A}^{\mathbf{T}}\mathbf{P}\mathbf{A}+\mathbf{S}\mathbf{S}^{\mathbf{T}}\right)^{-1}
    \mathbf{A}^{\mathbf{T}}\mathbf{P}\mathbf{A}\left(\mathbf{A}^{\mathbf{T}}\mathbf{P}\mathbf{A}+\mathbf{S}\mathbf{S}^{\mathbf{T}}\right)^{-1}
\end{equation}

Код также переходит к тестированию глобальной модели, чтобы проверить, адекватна
ли математическая модель или в данных есть отклонения. Если выполняется
следующее условие, то корректировку модели можно считать правильной и
завершенной до уровня значимости $\alpha$:
\begin{equation}
    \chi^2=\frac{\mathbf{V}^{\mathbf{T}}\mathbf{P}\mathbf{V}}{\sigma^2_{0i}}<\chi^2_c\left(1-\alpha;m\right)
\end{equation}
где $\sigma^2_{0i}$ -- априорная дисперсия единичного веса, а
$\chi^2_c\left(1-\alpha;m\right)$ -- критическое значение распределения
$\chi^2$, когда доверительный уровень равен $\alpha$, а степень свободы регулировки
равна $m$.
\begin{equation}
    \chi^2_c\left(1-\alpha;m\right)=m\left[\chi_{1-\alpha}\left(\frac{2}{9m}\right)^{\frac{1}{2}}+1-\left(\frac{2}{9m}\right)\right]^3
\end{equation}
и для $0<\alpha<0.5$
\begin{equation}
    \chi_{1-\alpha} = t - \frac{2.515517 + 0.802853t + 0.010328t^2}{1 + 1.432788t + 0.189269t^2 + 0.001308t^3}
\end{equation}
и
\begin{equation}
    t = \sqrt{2\ln \left(\frac{1}{\alpha}\right)}
\end{equation}