\chapter[Установка сторонних программ]{Установка сторонних программ}
\label{chap:installing_external_programs}

\pg{} предоставляет конструкцию для обработки данных микрогравиметрии. Таким
образом, вызовы внешних программ в \pg{} возможно легко включить. В настоящее
время могут быть вызваны две программы: функция \textbf{\textsf{PREDICT}} из
пакета \textbf{\textsf{ETERNA}} \cite{wenzel_1996} для расчета синтетических
приливов и \textbf{\textsf{MCGRAVI}} \cite{beilin_2006} для уравнивания сети.
Чтобы добавить любой другой вызов к внешним программам, лучший способ~--
просмотреть скрипты \pg{} и вдохновиться тем, как эти две программы
взаимодействуют. Здесь описана установка таких программ, если они будут
использоваться в \pg{}.

\section[Установка ETERNA]{Установка ETERNA}
\label{sec:eterna_installation}

Для пользователей Windows можно использовать функцию \verb|ETERNA| из пакета
\textbf{\textsf{ETERNA}} \cite{wenzel_1996} для поправок за прилив. Это особенно
подходит, когда доступны параметры приливов для конкретного места (например, с
помощью анализа приливов по данным сверхпроводящего гравиметра, поскольку это
также включает эффект океанической нагрузки). Каталог \pg{} включает в себя
облегченную версию \textbf{\textsf{ETERNA}} с минимумом, необходимым для
правильной работы функции \textbf{\textsf{PREDICT}}. Просто скопируйте папку
\verb|/eterna33| из папки \verb|main_code/external_files/| в корневой каталог
(\verb|C:\|). Когда \pg{} попросят запустить \verb|predict|, он скопирует
экземпляр файла \verb|predict.exe|, присутствующий в папке
\verb|main_code/external_files/|, в выходной каталог съёмки и запустите его. Эта
программа \verb|predict.exe| затем вызовет данные о приливном потенциале из
папки \verb|C:/eterna33|. Полный пакет \textbf{\textsf{ETERNA}} также доступен
бесплатно на сайте ICET: \url{http://www.upf.pf/ICET/soft/index.html}.

\section[Установка MCGRAVI]{Установка MCGRAVI}
\label{sec:mcgravi_installation}

\textbf{\textsf{MCGravi}} \cite{beilin_2006} может использоваться для поправки
за дрейф по методу наименьших квадратов и уравнивания сети в случае сложной сети
с несколькими известными абсолютными точками (инверсия методом наименьших
квадратов со взвешенными ограничениями, см. \cite{hwang_adjustment_2002}). В
качестве альтернативы, алгоритм инверсии наименьших квадратов без данных
\cite{hwang_adjustment_2002} закодирован в \pg{}.

\begin{itemize}
    \item Скопируйте папку \verb|MCGRAVI| в корневой каталог (\verb|C:\|) (или в другое место).

    \item Возможно потребуется перекомпилировать \verb|MCGRAVI|. В этом случае
    необходим g95. Его можно найти здесь:
    \url{http://www.g95.org/downloads.shtml}, или
    \url{http://math.hawaii.edu/~dale/190/fortran/fortran-windows-installation.html}, или
    \url{http://www.fortran.com/the-fortran-company-homepage/whats-new/g95-windows-
    download/}

    \item c Intel Fortran:
    \begin{itemize}
        \item Запустите \textbf{\textsf{Visual Studio}} и откройте проект \verb|Mc_gravi.vfproj|.
        Никаких конкретных параметров не требуется
        
    \end{itemize}

    \item с g95:
    \begin{itemize}
        \item \verb|cmd.exe| (или запустите \verb|cmd| в начальном меню), чтобы открыть
        консоль \verb|dos|. Перейдите в директорию \verb|mcgravi|
        \begin{lstlisting}
            cd /mcgravi 
        \end{lstlisting}
        
        Если необходимо
        \begin{lstlisting}
            make clean
        \end{lstlisting}
        или
        \begin{lstlisting}
            del *.o
            del *.mod
        \end{lstlisting}

        \begin{lstlisting}
            make all
        \end{lstlisting}

        \item чтобы запустить \textbf{\textsf{mcgravi}}, скопируйте
        \verb|mcgravi.exe| в рабочую директорию и \verb|conf.conf| в окне \verb|dos|
        
    \end{itemize}

    \item Добавьте путь к исполняемому файлу в переменную окружения \verb|PATH|.
    Перейдите в раздел \textbf{Advanced system parameters} на странице
    \textbf{System} (панель конфигурации). Выберите переменную окружения. На
    нижней панели (системные переменные) выберите строку \textbf{Path} и
    \textbf{Modify}.  Скопируйте/вставьте значение переменной в текстовый
    редактор (\textbf{Ctrl+A}/\textbf{Ctrl+C} $\rightarrow$ \textbf{Ctrl+V}). Добавьте путь, за
    которым следует ';'. Больше ничего не меняйте.

    \item Установите \textbf{\textsf{Perl}}
    
    \item Также для вывода карт из \textbf{\textsf{mcgravi}} требуется
    \textbf{\textsf{GMT}}, но все работает нормально, если он не установлен.
    
\end{itemize}