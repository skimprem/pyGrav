\chapter[Установка сторонних программ]{Установка сторонних программ}
\label{chap:installing_external_programs}

pyGrav предоставляет фрейм для обработки данных о микрогравитации. Таким
образом, легко возможно включить вызовы внешних программ в pyGrav. В настоящее
время могут быть вызваны две программы: функция ПРОГНОЗИРОВАНИЯ из пакета ETERNA
\cite{wenzel_1996} для расчета синтетических приливов и MCGRAVI \cite{beilin_2006} для
настройки сети. Чтобы добавить любой другой вызов к внешним программам, лучший
способ~-- просмотреть скрипты pyGrav и вдохновиться тем, как эти две программы
взаимодействуют. Здесь описана установка таких программ, если они будут
использоваться в pyGrav

\section[Установка ETERNA]{Установка ETERNA}
\label{sec:eterna_installation}

Для пользователей window можно использовать функцию ETERNA из пакета ETERNA
\cite{wenzel_1996} для коррекции приливов. Это особенно подходит, когда доступны
параметры приливов для конкретного участка (например, с помощью анализа приливов
по данным сверхпроводящего гравиметра, поскольку это также включает эффект
океанической нагрузки). Каталог pyGrav включает в себя облегченную версию ETERNA
с минимумом, необходимым для правильной работы функции ПРОГНОЗИРОВАНИЯ. Просто
скопируйте папку /eterna33 из папки \verb|main_code/external_files/| в корневой каталог
(\verb|C:\|). Когда pyGrav попросят запустить \verb|predict|, он скопирует экземпляр
\verb|predict.exe| файл, присутствующий в папке \verb|main_code/external_files/|, в выходной
каталог опроса и запустите его. Это \verb|predict.exe| затем программа вызовет данные о
приливном потенциале из \verb|C:/eterna33| папка. Полный пакет услуг ETERNA также
доступен бесплатно в кассе: \url{http://www.upf.pf/ICET/soft/index.html}.

\section[Установка MCGRAVI]{Установка MCGRAVI}
\label{sec:mcgravi_installation}

MCGravi (Beilin, 2006) может использоваться для корректировки дрейфа по методу
наименьших квадратов и компенсации сети в случае сложной сети с несколькими
известными абсолютными точками (взвешенная инверсия наименьших квадратов
ограничения, см. \cite{hwang_adjustment_2002}). В качестве альтернативы,
алгоритм инверсии наименьших квадратов без данных \cite{hwang_adjustment_2002}
закодирован в pyGrav.

\begin{itemize}
    \item Скопируйте папку MCGRAVI в корневой каталог (\verb|C:\|) (или в другое место).

    \item МАКГРОУ, возможно, мне потребуется перекомпилировать. В этом случае
    необходим G95. Его можно найти здесь:
    \url{http://www.g95.org/downloads.shtml}, или
    \url{http://math.hawaii.edu/~dale/190/fortran/fortran-windows-installation.html}, или
    \url{http://www.fortran.com/the-fortran-company-homepage/whats-new/g95-windows-
    download/}

    \item c intel Fortran:
    \begin{itemize}
        \item Запустите Visual Studio и откройте проект \verb|Mc_gravi.vfproj|.
        Никаких конкретных параметров не требуется
        
    \end{itemize}

    \item с g95:
    \begin{itemize}
        \item cmd.exe (или exec -> cmd в начальном меню), чтобы открыть
        консоль dos. компакт-диск для /mcgravi. ”make clean“, если необходимо,
        или ”del *.o” \& ”del *.mod"

        \item make all
        
        \item чтобы запустить mcgravie, скопируйте mcgravi.exe в рабочем каталоге и mcgravie conf.conf в окне dos
        
    \end{itemize}

    \item Добавьте путь к исполняемому файлу в переменную окружения PATH.
    Перейдите в раздел "Дополнительные параметры системы" на странице "Система"
    (панель конфигурации). Выберите переменную окружения. На нижней панели
    (системные переменные) выберите строку "Путь" и "изменить".
    Скопируйте/вставьте значение переменной в текстовый редактор (Ctrl+A/ Ctrl+C
    => Ctrl+V). Добавьте путь, за которым следует ‘;’. Больше ничего не меняйте.

    \item Установите Perl
    
    \item GMT также требуется для вывода карт из mcgravi, но все работает
    нормально, если он не установлен.
    
\end{itemize}