\chapter[Обзор]{Обзор}
\label{overview}

\pg{} \cite{hector_2016} разработан для обработки данных микрогравиметрии, и в
основном посвящен центраферным гравиметрическим исследованиям.  Пользователь
может выбрать между графическим пользовательским интерфейсом (GUI) или
классическими сценариями Python путём запуска функций \pg{} для обработки
данных: поправки за приливы и атмосферные явления, выбор данных и корректировку
дрейфа.  Открытый исходный код написан на Python~2.7 и соответствует
объектно-ориентированной схеме, которая позволяет быстро внедрять новые
функции/опции. В настоящее время для чтения доступен только формат файла
Scintrex CG5 ASCII, но любой другой формат может быть легко добавлен в процедуры
чтения \pg{}. Ниже кратко излагаются ключевые моменты:
\begin{itemize}
    \item Вместо управления различными конкретными программами с
    соответствующими форматами файлов ввода/вывода, дан единый интерфейс для
    различных этапов обработки (поправок, выбора данных, уравнивание дрейфа,
    двойные разности \dots),

    \item Обеспечен уникальный и простой в использовании интерфейс для выбора
    данных, как с графическим, так и с табличным отображением, а также
    автоматические критерии выбора для ускорения обработки.
    
    \item Пакет написан на языке Python~2.7 с открытым исходным кодом, и также
    может использоваться для переноса других программ (таких как MCGRAVI для
    компенсации сети, или ETERNA PREDICT для вычисления синтетических приливов).
    
    \item Пакет написан в объектно-ориентированном стиле, подходящем для
    микрогравиметрических данных, для которых четко определены объекты с
    определенными свойствами/функциями и соблюдена интуитивно понятная иерархия
    (гравиметрическая \textbf{кампания} с несколькими \textbf{съёмками}, каждый
    из которых состоит из разных \textbf{петель}, состоящих из нескольких
    \textbf{пунктов}).
    
    \item Структура кода (GUI также закодирован в объектно-ориентированном стиле
    с использованием PyQt) позволяет легко реализовать дополнительные функции
    (такие как формат ввода-вывода или взаимодействие с другими программами).

\end{itemize}

Данное руководство пользователя представлено следующим образом:
\begin{itemize}
    \item В \textbf{разделе~\ref{chap:quick_start}} предлагается \textbf{<<быстрый
    запуск>>} для запуска \pg{}. 
    
    \item \textbf{Функции} \pg{} подробно описаны в
    \textbf{разделе~\ref{chap:pygrav_functions}}.
    
    \item В \textbf{разделе~\ref{chap:test-case}} перечислены шаги для запуска
    предоставленных \textbf{тестовых данных}. Это лучший способ перейти к \pg{}.
    
    \item \textbf{Структура кода} кратко представлена в
    \textbf{разделе~\ref{chap:code_structure}} для тех, кто заинтересован в
    модификации сценариев на языке python
    
    \item В \textbf{разделе~\ref{chap:acquisition_protocol}} предложен \textbf{протокол
    по сбору данных}, основанный на опыте автора и позволяющий лучшем
    использовать текущие возможности \pg{}.
    
    \item
    \textbf{Раздел~\ref{chap:least-square_inversion_and_error_propagation}}
    напоминает формулы для \textbf{уравнивания по методу наименьших квадратов}, и
    распределение ошибок
    
    \item В \textbf{разделе~\ref{chap:installing_external_programs}} приведена
    процедура установки внешних программ, которые могут использоваться в \pg{} 
    (ETERNA и MCGRAVI).

\end{itemize}