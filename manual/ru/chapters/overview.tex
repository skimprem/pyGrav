\chapter[Обзор]{Обзор}
\label{overview}

\pg{} \cite{a10-063_new_instrument_data_report} предназначен для покадровых исследований
гравитации а также для обработки данных о микрогравитации Пользователь может
выбрать между графическим пользовательским интерфейсом (GUI) или классическими
скриптами на Python, вызвав функции pyGrav для обработки данных: поправки на
приливы и атмосферные явления, выбор данных и корректировку дрейфа. Код с
открытым исходным кодом, написан на Python 2.7 и соответствует
объектно-ориентированной схеме, которая позволяет быстро внедрять новые функции
/ опции. В настоящее время формат файла Scintrex CG5 ASCII доступен только для
чтения , но любой другой формат может быть легко добавлен в процедуры чтения
pyGrav. Ключевые моменты кратко излагаются ниже:
\begin{itemize}
    \item Обеспечьте единый интерфейс для различных этапов обработки
    (исправления, выбор данных, корректировка дрейфа, двойные различия ...),
    вместо того, чтобы беспокоиться об управлении различными конкретными
    программами с соответствующими форматами файлов ввода / вывода.
    
    \item Обеспечьте уникальный и простой в использовании интерфейс для выбора
    данных, как с графическим, так и с табличным отображением, а также
    автоматические критерии выбора для ускорения обработки.
    
    \item Написан на языке Python 2.7 с открытым исходным кодом, также может
    использоваться для переноса других программ (таких как MCGRAVI для
    компенсации сети или ETERNA PREDICT для вычисления синтетических приливов).
    
    \item Написан в объектно-ориентированном стиле, подходящем для данных о
    микрогравитации, для которых четко определены объекты с определенными
    свойствами / функциями и соблюдена интуитивно понятная иерархия
    (гравитационная кампания с несколькими обзорами, каждый из которых состоит
    из разных циклов, состоящих из нескольких станций).
    
    \item Структура кода (графический интерфейс также закодирован в
    объектно-ориентированном стиле с использованием PyQt) позволяет легко
    реализовать дополнительные функции (такие как формат ввода-вывода или
    взаимодействие с другими программами).

\end{itemize}