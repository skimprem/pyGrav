\chapter[Быстрый старт]{Быстрый старт}
\label{chap:quick_start}

\pg{}~-- это набор из четырех скриптов на Python с расширением \verb|.py|:
\verb|pyGrav_main.py|, \verb|data_objects.py|, \verb|model_Classes_tree_and_table.py|, и
\verb|synthetic_tides.py|. Основные функции для обработки микрогравиметрических данных
определены в \verb|data_objects.py|, а графический пользовательский интерфейс (GUI)
определен в \verb|pyGrav_main.py|. Третий сценарий, \verb|model_Classes_tree_and_table.py|,
содержит только функции для отображения и взаимодействия с данными на этапе
выбора данных \pg{}, в то время как четвертый скрипт, \verb|synthetic_tides.py| это
всего лишь модуль, в котором размещены функции для расчета синтетических
приливов. \textbf{Чтобы запустить пользовательский интерфейс \pg{}, запустите сценарий}
\verb|pyGrav_main.py|. Хороший способ запустить и/или отредактировать скрипт
на Python~-- использовать Spyder, визуальный интерфейс, аналогичный Matlab,
который предоставляет редактор сценариев и структуру кода, консоль \dots Оказавшись в
Spyder, откройте скрипт \verb|pyGrav_main.py| и запустите его с помощью
клавиши F5.

\begin{itemize}
    \item Для пользователей Windows или Mac хорошим вариантом является загрузка
    и установка Spyder вместе с обширным списком модулей Python. Хорошие
    варианты~-- Анаконда (\url{https://store.continuum.io/cshop/anaconda/}) или
    Python(x,y) (\url{https://code.google.com/p/pythonxy/}).

    \item Для пользователей Linux Spyder можно получить в менеджере пакетов.
    Также, если установлен Python и все необходимые модули, возможно напрямую
    запустить скрипт на Python с помощью команды \verb|python pyGrav_main.py| из
    каталога исходных файлов.

\end{itemize}

Для получения краткого обзора функционирования \pg{} следуйте руководству
тестовым данным из раздела~\ref{chap:test-case} данного руководства.