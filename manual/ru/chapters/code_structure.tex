\chapter[Структура кода]{Структура кода}
\label{chap:code_structure}

\section[Объектно-ориентированное программирование и данные микрогравиметрии]{Объектно-ориентированное программирование и данные микрогравиметрии}
\label{sec:object-oriented programming_and_microgravity_data}

Этот мощный способ программирования, для которого особенно подходит Python,
основан на наборе определений объектов (классов), которые содержат оба атрибута
(или свойства), такие как временной ряд, имя, словарь других объектов\dots и
методы (или функции/определения), которые описывают, что может делать объект.
Таким образом, как только программист привыкает к таким концепциям, этот метод
программирования не является линейным, но способствует удобочитаемости кода.
Метод, например, для записи новых выходных форматов, может быть легко добавлен в
соответствующий объект и вызван из основной программы, добавив всего несколько
строк и сохранив общий фрейм исходного кода.

Это особенно подходит для данных микрогравиметрии, главным образом потому, что
данные могут быть организованы в виде структур (объектов) в соответствии с
иерархическим определением (кампания, которая включает в себя несколько съёмок,
которые включают в себя несколько петель, которые включают в себя несколько
пунктов, которые включают временные ряды полученных данных). Объекты могут быть
физически идентифицированы (кампания, съёмка, петля, пункт), а также связанные с
ними методы (чтение ascii-файла CG5 должно быть определено в самом широком
объекте -- кампании; запись входного файла \textbf{\textsf{mcgravi}} -– файла
'c' -- для корректировки дрейфа должна быть определена в объекте петли, в то
время как корректировка дрейфа должна вызываться или основываться на объекте
съемки; выбор данных будет в основном касаться пунктов). Эта логическая иерархия
позволяет хранить данные в виде структур.

\section[Графический интерфейс и PyQt]{Графический интерфейс и PyQt}
\label{sec:gui_and_pyqt}

\textbf{\textsf{PyQt}} использует мощный язык программирования Python и его
пригодность для объектно-ориентированного программирования вместе с библиотеками
\textbf{\textsf{Qt GUI}} (графический пользовательский интерфейс).
\textbf{\textsf{PyQt}} состоит из нескольких модулей, таких как фундаментальный
\textbf{\textsf{QtCore}} (для функций, отличных от GUI, таких как использование
файлов, потоков, процессов или времени) и \textbf{\textsf{QtGui}} (для
графических компонентов), среди которых определены от десятков до сотен классов,
содержащих многочисленные функции и свойства.

\section[Код pyGrav]{Код \pg{}}
\label{sec:pygrav_code}

Код основан на объектно-ориентированном программировании (ООП). Это можно
рассматривать как два параллельных набора определений:
\begin{itemize}
    \item Функции хранения данных и манипулирования ими: ядро программы, где
    определены классы и функции для операций с данными. Это тот самый
    \verb|data_object.py| исходный файл.

    \item Функции GUI: функции графического интерфейса пользователя, которые
    связывают требования пользователя с операциями с данными, определенными в
    предыдущем наборе определений. Это тот самый \verb|pyGrav_main.py| исходный
    файл.
    
\end{itemize}

В целом, код обильно прокомментирован, и описан каждый класс, подкласс, функции и свойства.

\subsection[Объектный файл данных]{Объектный файл данных}
\label{subsec:data_object_file}

Этот модуль содержит основные классы программы:
\begin{itemize}
    \item Базовый класс -- это объект типа \verb|ChannelList|, который в основном
    содержит списки каналов, подобные тем, что содержатся в выходных файлах CG5
    ascii (сила тяжести, наклоны, температура, стандартное отклонение, время
    \dots).

    Производные классы следуют логической иерархии, где каждый <<подкласс>>
    создается как элемент словаря из родительского класса:
    \begin{itemize}
        \item класс \verb|Campaign| содержит словарь объектов типа \verb|Survey|
        \begin{itemize}
            \item класс \verb|Survey| содержит словарь объектов типа \verb|Loop|
            \begin{itemize}
                \item класс \verb|Loop| содержит словарь объектов типа \verb|Station|
                % \begin{itemize}
                    \item класс \verb|Station| o содержит временные ряды для каждого пункта 
                % \end{itemize}
            \end{itemize}
        \end{itemize}
    \end{itemize}
    
\end{itemize}

Каждый из этих объектов является производным от объекта списка каналов.
Экземпляр базового класса \verb|Campaign| содержит весь набор данных. У каждого
класса также есть определенные свойства и функции заполнения, записи, манипуляции
и обработки, вызываемые из основной программы (рис.~\ref{fig:example_of_pygrav_snapshots}).

\begin{figure}
    \includegraphics[width=\textwidth]{figures/pygrav_chart_and_structures_imbrications}
    \caption{Диаграмма \pg{} и обозначения структур (объектов): в объекте
    \textbf{Campaign} есть словарь из нескольких \textbf{Survey}. В объекте \textbf{Survey} есть словарь
    из нескольких \textbf{Loop}. В объекте \textbf{Loop} есть словарь из нескольких \textbf{Station}.
    Каждый из этих объектов является производным от объекта \textbf{ChannelList} (т.е.
    они содержат несколько временных рядов).}
    \label{fig:pygrav_chart_and_structures_imbrications}
\end{figure}

\subsection[Файл графического интерфейса]{Файл графического интерфейса}
\label{subsec:gui_file}

Это файл, который должен быть выполнен для запуска \pg{}. Он содержит
единственный класс под названием \verb|mainProg|, который является объектом
\verb|QMainWindow|, производным от \verb|Qt|. Наиболее важными свойствами класса
\verb|mainProg| являются объект \verb|Campaign|, который содержит весь набор
данных, а также каталоги данных и выходных данных. Большинство функций класса
\verb|mainProg| связывают пользовательский интерфейс (определенный в функциях) с
кодом обработки, записанным в файле объекта данных, для изменения состояний
объекта \verb|Campaign| (набора данных).

\section*{Ссылки}

Несколько ссылок для тех, кто заинтересован в изменении кода
\begin{itemize}
    \item 
    Несколько руководств по Python:

    \url{https://docs.python.org/2/tutorial/} Руководство для Python версии 2.7

    \url{http://zetcode.com/lang/python/}

    \url{http://marvin.cs.uidaho.edu/Teaching/CS515/pythonTutorial.pdf} (Руководство G. Van Rossum)

    \item
    Руководство PyQt:

    \url{http://zetcode.com/gui/pyqt4/}

    \item
    Ссылки на класс PyQt:

    \url{http://pyqt.sourceforge.net/Docs/PyQt4/classes.html}

    \item
    Демо графиков PyQt:

    \url{http://eli.thegreenplace.net/2009/05/23/more-pyqt-plotting-demos/}

    \item
    Руководства программирования вида модели:

    \url{http://www.yasinuludag.com/blog/?p=98}

    \item
    Руководство Matplotlib:

    \url{http://web.archive.org/web/20100830233506/http://matplotlib.sourceforge.net/leftwich_tut.txt}
\end{itemize}